\section{Conclusion and Discussion}

In this work, we introduced an agentic framework for abductive event reasoning, demonstrating that separating evidence verification from causal inference yields a stable, conservative, reasoning engine. Our results on the SemEval 2026 Task 12 dataset highlight that reasoning in LLMs is an emergent property heavily dependent on architectural context.

A primary finding of this research is the emergent "Safety Bias" of the decoupled architecture. While the system frequently under-generated answers in multi-cause scenarios, often defaulting to single-choice formats, it maintained high semantic accuracy on the options it did select. This behavior suggests that our model act as risk-minimizers, preferring high-confidence partial proofs over the noise of exhaustive retrieval. However, this cautious approach backfired when the answer was 'None of the Other.' In these cases, the model struggled to admit that no cause existed. Instead of accepting that none of the options worked, it often forced a connection where there wasn't one, effectively guessing rather than staying silent.

Finally, our comparative analysis with the graph-based baseline (CausalRAG) challenges the assumption that structural complexity intrinsically yields better performance. Despite its theoretical benefits, CausalRAG failed to outperform the text-based baseline in our experiments. We attribute this to ``contextual drift,'' where spurious connections within the automatically generated graphs distracted the model rather than reinforcing the core evidence. This finding suggests that navigating dense, noisy graph structures may require agents with significantly higher capacity than the 7B-parameter model employed here. While a larger model might effectively filter this structural noise, our results indicate that for efficient, mid-sized agents, the \textbf{Hybrid Retrieval Strategy coupled with the Multi-Agent architecture} remains the superior choice, offering robustness without the excessive cognitive overhead required to parse imperfect graphs.